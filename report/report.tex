\documentclass[a4paper,12pt]{article}

\usepackage{mystyle}

\lstset{
    language=Oz,
    basicstyle=\footnotesize,
    %title=\lstname,
    frame=single,
    numbers=left,
    breaklines=true,
}

\begin{document}

\begin{center}
\begin{tabu} to \textwidth {lX[c]r}
    Xavier Lambein & \large{\textbf{Rapport: DJ'Oz}} & Victor Lecomte \\
    54621300 & LFSAB1402 & 65531300 \\
    \hline
\end{tabu}
\end{center}

\section{Structure et Conception}
\label{sec:struct}

\subsection{Conception générale}

Pour plus de clareté, nous avons choisi de diviser notre code source en quatre fichiers différents, correspondant à quatre fonctions ou aspects particuliers :
\begin{itemize}
    \item \texttt{code.oz:} rassemble les autres fichers \texttt{.oz} et exécute \texttt{Project.run};
    \item \texttt{interprete.oz:} définit la fonction \texttt{Interprete}, servant à transformer une partition en liste d'échantillons ;
    \item \texttt{mix.oz:} définit la fonction \texttt{Mix}, servant à transformer une liste d'échantillons et d'effets en vecteur audio ;
    \item \texttt{envelopes.oz:} définit les différentes enveloppes que le programme propose d'appliquer aux échantillons (plus de détails sur les enveloppes dans la section \ref{sec:enveloppes}).
\end{itemize}

Comme demandé par l'énoncé du projet, le code que nous avons écrit utilise exclusivement le paradigme fonctionnel. Aucune entorse n'a été nécessaire pour implémenter des fonctions qui nous semblaient à la fois simple et efficaces. Cependant, notre programme ne présente aucune gestion des erreurs: sans exceptions, celles-ci auraient diminué significativement la clareté de notre implémentation.

\subsection{Fonction \texttt{Interprete}}

Le coeur de la fonction \texttt{Interprete} est la sous-fonction \texttt{InterpreteRecursive}. Celle-ci analyse une partition et produit un vecteur d'échantillons. Son corps n'est rien d'autre qu'un grand \texttt{case} qui traite les suites de partitions (par récursion), les transformations et les notes.

Les paramètres de \texttt{InterpreteRecursive} sont les suivants:
\begin{itemize}
    \item \texttt{Score}: une partition à analyser;
    \item \texttt{Mod}: la composée des transformations à appliquer (voir ci-dessous);
    \item \texttt{Next}: une liste d'échantillons à ajouter à la fin de la liste produite par \texttt{Score} (détails ci-dessous).
\end{itemize}

Les transformations sont effectuées par \emph{higher-order programming}: lorsqu'une certaine transformation est traitée par \texttt{InterpreteRecursive}, elle produit une fonction composable à partir des paramètres de la transformation via une fonction \texttt{Make\textit{NomDeLaTransformation}}.

Ensuite, la sous-partition sur laquelle s'applique la transformation est à son tour traitée via \texttt{InterpreteRecursive}; cependant, la fonction composable produite ci-dessus est passée dans l'argument \texttt{Mod} afin de \og{}Diffuser\fg{} la transformation aux sous-partitions. Dans le cas où \texttt{Mod} aurait déjà été définie plus haut, la nouvelle transformation est composée aux précédentes.

Enfin, lorsque \texttt{InterpreteRecursive} en arrive à traiter une note, il la convertit en échantillon et applique sur celui-ci la fonction \texttt{Mod}, c'est-à-dire la composée des transformations.

Afin d'obtenir une liste unique de tous les échantillons produits par des appels récursifs successifs, nous avons recours au paramètre \texttt{Next} de \texttt{InterpreteRecursive}. Voilà comment nous aurions procédé sans: lorsqu'une liste est traitée dans le \texttt{case}, il est nécessaire de faire un appel récursive à la tête de liste, puis à la queue, puis d'utiliser \texttt{Append} sur les deux listes d'échantillons produites afin d'obtenir une unique liste continue ; cela revient à remplacer \texttt{nil} à la fin de la première liste par la seconde liste.

Cependant, étant donné que nous parcourons de toute façon ces listes au moins une fois pour créer les échantillons, plutôt que d'utiliser \texttt{Append} pour les combiner, nous passons, lors de l'appel à \texttt{InterpreteRecursive} sur la tête, un argument \texttt{Next} qui contient la liste d'échantillons produits par l'appel à \texttt{InterpreteRecursive} sur la queue. Ensuite, à l'intérieur de \texttt{InterpreteRecursive}, plutôt que de terminer une liste d'échantillons par \texttt{|nil}, nous la terminons par \texttt{|Next}. Ainsi, les listes sont combinées sans allonger le temps d'exécution par des appels à \texttt{Append}. Cette astuce nous permet de garder une complexité linéaire.

\subsection{Fonction Mix}
\label{sec:mix}

Bien que plus compliquée, la fonction \texttt{Mix} fonctionne sur le même principe que \texttt{Interprete}: il s'agit d'un \texttt{case} sur une musique/liste de morceaux (fonction \texttt{MusicToAV}). Chaque morceau est analysé dans un autre \texttt{case} et traité différemment selon qu'il s'agit d'une voix, d'une partition, d'un fichier wave, d'un merge ou d'un filtre (fonction \texttt{PieceToAV}). De nouveau, la plupart des sous-fonctions de \texttt{Mix} utilisent un paramètre \texttt{Next}, pour les mêmes raisons que dans \texttt{InterpreteRecursive}.

L'application des filtres, cependant, ne se fait absolument de la même manière que les transformations dans \texttt{InterpreteRecursive}. La raison est qu'un filtre nécessite généralement d'appliquer un traitement sur tout un vecteur audio à la fois, et non pas élément par élément. Nous avons donc une fonction définie pour chaque filtre, à laquelle nous fournissons un vecteur audio se terminant par \texttt{nil} et le vecteur suivant qui doit venir dans la musique (le paramètre \texttt{Next}). Le filtre applique son traitement sur le vecteur tout entier en le parcourant puis retourne un nouveau vecteur audio.

Contrairement aux transformations, les fonctions de filtre sont généralement très différentes les unes des autres, mais elles fonctionnent toutes selon un même squelette: lors d'un appel récursif au filtre effectué sur tout le vecteur, une transformation est appliquée à chaque élément du vecteur. Cette transformation dépend de paramètres déterminés par le vecteur dans son entièreté, ou simplement par les éléments précédents ; c'est la raison pour laquelle les filtres ne fonctionnent pas comme les transformations. Enfin, chaque élement traité est ajouté ajouté à une nouvelle liste, terminant non plus par \texttt{nil} mais par \texttt{Next}.

Pour terminer, abordons la partie plus \og{}physique\fg{} de \texttt{Mix}: la génération d'un vecteur audio à partir d'échantillons (fonction \texttt{SampleToAV}). La fonction agit de trois façons différentes selon que l'échantillon est un silence, une note pure ou une note d'un instrument. Dans le cas d'un silence, la elle génère simplement un vecteur de zéros de la longueur voulue. Dans le cas d'une note pure, une sinusoïde de la fréquence de l'échantillon est produite, puis une enveloppe est appliquée dessus (voir sections \ref {sec:impl_enveloppes} et \ref{sec:enveloppes} pour plus de détails sur les enveloppes). Enfin, si un instrument est donné dans l'échantillon, plutôt que de générer une sinusoïde, le programme utilise le nom de l'instrument et la hauteur de la note pour récupérer le fichier wav correspondant dans \texttt{./wave/instruments/} et le convertir en vecteur audio ; ensuite, le vecteur est tronqué et un fondu en fermeture est appliqué dessus.

\subsection{Implémentation des enveloppes}
\label{sec:impl_enveloppes}

Dans notre programme, nous utilisons des enveloppes sonores pour ajuster
le volume des échantillons au début et à la fin de ceux-ci.
Nous avons donc défini les enveloppes comme des fonctions qui prennent
une position dans l'échantillon et renvoie un facteur par lequel il faut
multiplier l'intensité sonore.

Puisque les enveloppes ainsi définies ont besoin de plus d'informations que ce
qui leur est donné en paramètre, nous avons utilisé leur environnement
contextuel pour mémoriser leurs paramètres, la durée de l'échantillon concerné,
et pour le cas de l'enveloppe hyperbolique (voir section~\ref{sec:enveloppes}),
un facteur adaptatif calculé à partir de la durée et de l'attaque.

Prenons l'exemple de l'enveloppe trapézoïdale:
\begin{lstlisting}
fun {EnvTrapezoid Att Rel Dur}
    fun {$ Pos}
        if Pos < Att then
            Pos/Att
        elseif Dur-Pos < Rel then
            (Dur-Pos)/Rel
        else
            1.0
        end
    end
end
\end{lstlisting}

La fonction \texttt{EnvTrapezoid} renvoie une enveloppe:
une fonction qui pour une position donne un facteur de volume
(en forme de trapèze).
Même si elle a besoin de connaître l'attaque \texttt{Att} et la durée
\texttt{Dur} de l'échantillon, ces données ne sont pas des arguments de la
fonction.

Cette fabrique de fonctions et les deux autres se trouvent dans le fichier
\texttt{envelope.oz}.

Nous utilisons les enveloppes ainsi générées dans deux contextes:
\begin{itemize}
    \item \textbf{Générateur de sinusoïdes:}
        la fonction \texttt{SinusoidToAV} prend en paramètre (en plus de la
        hauteur et la durée) une enveloppe, qu'il applique à la sinusoïde
        qu'elle génère. Cette enveloppe est choisie dans \texttt{SampleToAV},
        selon l'instrument entré:
\begin{lstlisting}
% Default instrument is a sinusoid with ADSR envelope.
of none then
    {SinusoidToAV P Dur {EnvADSR 0.03 0.01 0.8 0.03 Dur} Next}

% If the instrument only specifies the envelope to use.
[] trapezoid(att:A rel:R) then
    {SinusoidToAV P Dur {EnvTrapezoid A R Dur} Next}
[] adsr(att:A dec:D sus:S rel:R) then
    {SinusoidToAV P Dur {EnvADSR A D S R Dur} Next}
[] hyperbola(att:A) then
    {SinusoidToAV P Dur {EnvHyperbola A Dur} Next}
\end{lstlisting}
        si l'instrument est \texttt{none} un filtre ADSR par défaut est
        passé à \texttt{SinusoidToAV}, mais si l'instrument indique
        l'utilisation d'une enveloppe spécifique
        (\texttt{trapezoid(...)}, \texttt{adsr(...)}, \texttt{hyperbola(...)}),
        alors l'enveloppe correspondante est construite et passée.
    \item \textbf{Extraits audio d'instruments:}
        une fois un extrait d'instrument découpé à la bonne taille il faut
        adoucir la fin de l'extrait, pour éviter un saut auditif (le début ne
        pose pas problème parce qu'il est déjà lisse).
        Dès lors, nous choisissons un filtre trapézoïde qui lisse uniquement la
        fin:
\begin{lstlisting}
% Otherwise we have to read the instrument wav file.
else
    RawAV = {ReadInstrumentWav I P}
    
    % We apply a trapezoid filter to smooth the end (the start is
    % already smooth).
    Env = {EnvTrapezoid 0.0 0.03 Dur}
    fun {ApplyEnv Pos AV}
        case AV
        of H|T then
            H*{Env Pos}|{ApplyEnv Pos+Step T}
        else
            Next
        end
    end
in
    % We cut the instrument sample to the specified duration then we
    % apply the envelope.
    {ApplyEnv 0.0 {Cut 0.0 0.0 Dur RawAV nil}}
\end{lstlisting}
        Il s'agit ensuite de reparcourir le vecteur audio pour appliquer le
        filtre sur ses éléments, ce que fait la sous-fonction \texttt{ApplyEnv}.
\end{itemize}

\subsection{Complexité}

Notre code ne présente jamais de boucles imbriquées sur des listes d'échantillons ou de vecteurs audio, et ce notamment grâce à l'usage de \texttt{Next}. Nous avons donc une complexité linéaire. 

\section{Limitations}

Lors de l'implémentation du projet, trois limitations nous sont rapidement apparues: le format des partitions, la lenteur et le manque de robustesse.

Le premier obstacle rencontré lors de l'écriture d'une musique pour le projet est le format dans lequel les partitions sont écrites: trop verbeux et pas assez proche de la façon dont on écrit de la musique, il demande de consacrer beaucoup de temps à écrire et à réorganiser le code de la partition au moindre ajout ou changement, ce qui impacte fortement la créativité du compositeur en herbe. Une manière de résoudre ce problème aurait été d'implémenter une syntaxe alternative pour l'écriture des partitions (par exemple inspirée de LilyPond\footnotemark), mais le temps nous a manqué.
\footnotetext{\url{http://www.lilypond.org/index.fr.html}}

Ensuite, et il s'agit sans doute du plus grand obstacle, le programme, à cause de Mozart 2, est incroyablement lent: une musique de quelques notes prends plusieurs dizaines de secondes à \og{}compiler\fg{}.

Enfin, le dernier point n'est pas un obstacle à l'utilisateur mais est néanmoins très important: notre programme ne présente aucune gestion des erreurs. Un \emph{input} erronné n'est donc jamais susceptible de produire un message d'erreur clair et adapté au problème; au lieu de cela, seul les erreurs de Oz sont affichées et, souvent, elles n'offrent que peu d'indications quant à l'origine du problème. Cet aspect aurait pu être adressé en utilisant des exceptions, au prix d'obtenir un code qui ne respecte plus totalement le paradigme fonctionnel.

\section{Extensions}

Comme extensions, nous avons choisi d'ajouter des enveloppes sonores
aux échantillons, la gestion des instruments, et une composition personnelle.

\subsection{Enveloppes sonores}
\label{sec:enveloppes}
Afin de d'éviter des sauts désagréables entre les notes,
et d'obtenir une qualité sonore plus lisse,
nous avons décidé d'utiliser des enveloppes sonores,
qui adoucissent le début et la fin des échantillons.

Dans le code, nous avons défini ces enveloppes comme des fonctions
qui renvoient un facteur de volume pour chacune des positions
de l'échantillon.
Plus de détails sur l'implémentation se trouvent
dans la section~\ref{sec:struct}.

Nous avons essayé plusieurs types d'enveloppes:
\begin{itemize}
    \item \textbf{Enveloppe en trapèze:} cette enveloppe prend deux paramètres:
        \emph{attack} et \emph{release}.
        
        Au début de l'échantillon, le volume va augmenter linéairement
        pendant un temps \emph{attack} avant d'atteindre son niveau de régime,
        puis à la fin il va diminuer linéairement pendant un temps
        \emph{release} jusqu'à un volume nul.
        
        Cette enveloppe est simple et permet de se débarasser du bruit de
        coupure, mais elle est assez peu paramétrable.
        Elle est implémentée par la fonction \texttt{EnvTrapezoid}.
    
    \item \textbf{Enveloppe ADSR:} analogue à la méthode du trapèze,
        elle accepte les quatre paramètres \emph{attack}, \emph{decay},
        \emph{sustain} et \emph{release}.
        
        Comme pour le trapèze, au début de l'échantillon, le volume augmente
        linéairement pendant un temps \emph{attack} jusqu'à atteindre un volume
        maximal. Mais après, le volume décroit de nouveau linéairement,
        pendant un temps \emph{decay}, avant d'attendre le niveau de régime
        \emph{sustain}, inférieur au niveau maximal.
        À la fin, le volume va diminuer en un temps \emph{release} comme pour
        le trapèze.
        
        Cette enveloppe a plus de liberté dans l'attaque initiale, ce qui
        permet d'imiter avec plus de précision des instruments de musique.
        Elle est implémentée par la fonction \texttt{EnvADSR}.
    
    \item \textbf{Enveloppe en hyperbole:}
        Cette méthode est notre invention.
        Nous trouvions que les autres méthodes n'étaient pas assez souples:
        en effet, pour qu'elles puissent gérer
        les échantillons courts sans saut,
        il faut que les paramètres \emph{attack} et \emph{release}
        soient petits, ce qui implique un son assez dur,
        même sur les échantillons plus longs.
        
        Pour éviter cela, il nous fallait donc une méthode qui s'adapte
        à la taille de l'échantillon, tout en gardant une dureté équivalente
        quelle qu'en soit la taille.
        Nous avons donc décidé de modéliser l'enveloppe comme un produit
        de deux hyperboles:
        \begin{itemize}
            \item l'une croissante,
                fixée à 0 au début et tendant vers 1 à la fin,
            \item l'autre décroissante,
                fixée à 0 à la fin et tendant vers 1 au début.
        \end{itemize}
        Elle s'exprime donc comme:
        \[
            \mbox{Volume} = \frac{x}{x-\mbox{att}} \times
                \frac{L-x}{L-x+\mbox{att}}
        \]
        où $x$ est la position, $L$ la longueur de l'échantillon et
        att un temps d'attaque.
        
        Le grand avantage de cette enveloppe est que le domaine n'est pas
        séparé en sous-intervalles, ce qui évite des sauts non-intentionnels.
        
        Un effet secondaire de cette méthode a été de diminuer le volume des
        courts échantillons quand leur longueur est trop proche du paramètre
        att.
        Nous avons résolu cela en multipliant l'expression par un facteur
        qui assure que l'amplitude au milieu de l'échantillon vaille 1.
        
        C'est l'enveloppe que nous avons utilisé pour adoucir nos propres sons.
        Elle est implémentée par la fonction \texttt{EnvHyperbola}.
\end{itemize}

\subsection{Instruments}

L'usage d'instruments externes permet d'obtenir facilement des sons intéressants qu'il aurait été compliqué de générer avec de simples sinusoïdes. Ils apportent une diversité de la plus haute importance à toute composition en MAO\footnote{Musique Assistée par Ordinateur} qui se respecte.

La gestion des instruments est effectuée exactement telle que spécifée dans l'énoncé du projet, c'est-à-dire dans le champ \texttt{instrument} des échantillons. Pour appliquer un instrument à une série de notes, on utilise \texttt{instrument(nom:$\left<instrument\right>$ $\left<partition\right>$)}. Les fichiers correspondant aux notes sont alors cherchés dans le dossier \texttt{./wave/instruments/}, sous le nom \texttt{\textit{instrument}\_\textit{note}.wav}, avec \texttt{\textit{note}} sous la forme \texttt{a4} ou \texttt{a4\#} selon qu'elle présente une alteration ou non.

Des détails supplémentaires sur l'implémentation des instruments sont donnés dans la section \ref{sec:mix}.

\subsection{Composition}

\end{document}













